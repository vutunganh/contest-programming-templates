\documentclass[10pt, a4paper, twoside]{article}
\usepackage[shorthands=off, czech]{babel}
\usepackage[utf8]{inputenc}
\usepackage[unicode]{hyperref}
\hypersetup{
  colorlinks=true,
  linkcolor=blue,
  linktoc=all
}
\usepackage{listings}
\usepackage{inconsolata}

% nastavuje code blocks
\lstset{
  basicstyle=\footnotesize,
  language=C++,
  frame=tb,
  numbers=left,
  breaklines
}

\title{Knihovna týmu My z Kačerova}
\author{Matyáš Křišťan, Jan Uhlík, Vu Tung Anh}

\begin{document}

\tableofcontents

\section{Algoritmy}
\subsection{Grafové algoritmy}

\subsubsection{Topsort}
O(n + m)
\lstinputlisting[firstline = 21, lastline = 42]{../Matyas/Templates/algorithms/graph.cpp}

\subsubsection{Floyd Warshall}
O(n**3)
\lstinputlisting[firstline = 44, lastline =61]{../Matyas/Templates/algorithms/graph.cpp}

\subsubsection{Maximální párování}
\lstinputlisting{../Tung/hopcroft-karp.cpp}

\subsubsection{Dinicův algoritmus}
Maximální tok O(V**2 * E)
\lstinputlisting[firstline = 22]{../Tung/dinic.cpp}

\subsubsection{Dijkstra}
\lstinputlisting[firstline = 24, lastline = 54]{../Tung/dijkstra.cpp}

\subsubsection{Bellman-ford}
\lstinputlisting[firstline=22, lastline=56]{../Tung/bellmanford.cpp}

\subsection{Stromové algoritmy}
\subsubsection{LCA}
O(n * log(n))
\lstinputlisting[firstline = 66, lastline = 98]{../Matyas/Templates/algorithms/graph.cpp}

\subsubsection{Nejdelší cesta ve stromě}
O(těžko říct, s každým dalším query se zlepšuje)
\lstinputlisting[firstline = 100, lastline = 129]{../Matyas/Templates/algorithms/graph.cpp}

\subsubsection{Střed stromu}
O(n)
\lstinputlisting[firstline = 132, lastline = 167]{../Matyas/Templates/algorithms/graph.cpp}


\subsection{Stringové algoritmy}
\subsubsection{Longest common substring}
O(n * m)
Jenom délka.
\lstinputlisting[firstline = 11, lastline = 22]{../Matyas/Templates/algorithms/string.cpp}

\subsubsection{Longest common subsequence}
O(n * m)
Jenom délka.
\lstinputlisting[firstline = 25, lastline = 38]{../Matyas/Templates/algorithms/string.cpp}

\subsubsection{Levenshtein}
\lstinputlisting{../Tung/levenshtein.cpp}

\subsection{Mergesort s inverzemi}
O(n * log(n)) asi
\lstinputlisting[firstline = 15, lastline = 57]{../Matyas/Templates/algorithms/mergesort.cpp}

\subsection{Binární vyhledávání}
\lstinputlisting[firstline = 7]{../Matyas/Templates/algorithms/binary-search.cpp}

\subsection{Nejkratší úsek pole se součtem roven k}
Pouze pro kladné prvky pole
\lstinputlisting[firstline = 12, lastline = 34]{../Matyas/Templates/algorithms/array.cpp}

\newpage

\section{Datové struktury}

\subsection{Big Integer}
\lstinputlisting[firstline = 9]{../Matyas/Templates/math/big-int.cpp}

\subsection{Fenwick}
JEŠTĚ JEDNOU ZDŮRAZNÍM, ŽE SE NEPOUŽÍVÁ T[0]\\
Chcete-li použít Fenwicka jako ``segmenťák'' (tedy update je add(l,d)+add(r+1,-d)), pak query stačí jako pref\_sum(i) a není třeba odečítat i-1.
\lstinputlisting{../Tung/fenwick.cpp}

\subsection{Union Find}
\lstinputlisting[firstline = 4]{../Matyas/Templates/data-structures/union-find.cpp}

\newpage

\section{Matematika}

\subsection{Jak správně generovat náhodná čísla?}
\begin{lstlisting}
  random_device r;
  mt19937 e(r()); // nebo mt19937_64 pro 64 bitova cisla
  uniform_int_distribution<int> uniform_dist(0,100000) // pravy kraj inclusive (kdybychom chteli generovat az intmax)
  uniform_dist(e); // tohle vraci random cislo
\end{lstlisting}

\subsection{Struktury}
\subsubsection{Matice}
\lstinputlisting[firstline = 13, lastline = 43]{../Matyas/Templates/math/common.cpp}

\subsection{GCD}
\lstinputlisting[firstline = 46, lastline = 60]{../Matyas/Templates/math/common.cpp}

\subsection{Rozšířený Euklidův algoritmus}
\lstinputlisting[firstline = 22, lastline = 26]{../Tung/eea.cpp}

\subsection{Kombinačni číslo}
\lstinputlisting[firstline = 63, lastline = 77]{../Matyas/Templates/math/common.cpp}

\subsection{Modularní inverze}
\lstinputlisting[firstline = 80, lastline = 114]{../Matyas/Templates/math/common.cpp}

\subsection{Kombinační číslo modulené(?)}
\lstinputlisting[firstline = 117, lastline = 130]{../Matyas/Templates/math/common.cpp}

\subsection{Eratosthenova síta}
\subsubsection{Nějaký basic}
O(n**3/2)
\lstinputlisting[firstline = 134, lastline = 147]{../Matyas/Templates/math/common.cpp}

\subsubsection{Trochu rychlejší}
O(n * log(n))
\lstinputlisting[firstline = 151, lastline = 167]{../Matyas/Templates/math/common.cpp}

\subsection{Faktorizace}
O(sqrt(n)), vrátí jako vektor párů {prvočíslo, počet}
\lstinputlisting[firstline = 173, lastline = 196]{../Matyas/Templates/math/common.cpp}

\subsection{Počet čtvercových dělitelů}
\lstinputlisting[firstline = 200, lastline = 220]{../Matyas/Templates/math/common.cpp}

\subsection{Rychlé umocňování čísel}
\lstinputlisting[firstline = 238, lastline = 248]{../Matyas/Templates/math/common.cpp}
\lstinputlisting[firstline = 251, lastline = 265]{../Matyas/Templates/math/common.cpp}

\subsection{Rychlé umocňování matic}
\lstinputlisting[firstline = 224, lastline = 234]{../Matyas/Templates/math/common.cpp}

\subsection{Rozšířené Eratosthenovo síto}
\lstinputlisting[firstline = 269, lastline = 285]{../Matyas/Templates/math/common.cpp}

\subsection{Rychlá faktorizace}
Potřeba rozšiřené Eratosthenovo síto výše
O(log(n))
\lstinputlisting[firstline = 288, lastline = 301]{../Matyas/Templates/math/common.cpp}

\subsection{Fast divisors}
At je to cokoliv
\lstinputlisting[firstline = 305, lastline = 326]{../Matyas/Templates/math/common.cpp}

\subsection{Odčítání v modulu}
Proprietary (x - y) \% k
\lstinputlisting[firstline = 329, lastline = 331]{../Matyas/Templates/math/common.cpp}

\subsection{Ternární vyhledávání}
O(log(n))
\lstinputlisting{../Tung/ternary-search.cpp}

\newpage

\section{Geometrie}
\lstinputlisting{../Tung/geometry/template.cpp}

\end{document}
