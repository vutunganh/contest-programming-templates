\documentclass{article}
\usepackage[shorthands=off, czech]{babel}
\usepackage[utf8]{inputenc}
\usepackage{hyperref}
\hypersetup{
  colorlinks=true,
  linkcolor=blue,
  linktoc=all
}
\usepackage{listings}
\usepackage{inconsolata}

% nastavuje code blocks
\lstset{
  basicstyle=\footnotesize,
  language=C++,
  numbers=none
}

\title{Knihovna týmu My z Kačerova}
\author{Matyáš Křišťan, Vu Tung Anh}

\begin{document}

\tableofcontents

\section{Algoritmy}
\subsection{Grafové algoritmy}
\subsubsection{Reprezentace}
\begin{lstlisting}
struct Vertex {
  vector<int> e;
  int deg_in;
  bool v = false;
  int depth;
  vector<P> query;
};
\end{lstlisting}

\subsubsection{Topsort}
\begin{lstlisting}
  vector<int> topsort(vector<Vertex> & g) {
    vector<int> res;
    vector<int> s;
    for (int i = 0; i < g.size(); ++i) {
      if (g[i].deg_in == 0) s.push_back(i);
    }
    while (!s.empty()) {
      const int cur = s.back();
      s.pop_back();
      res.push_back(cur);

      for (int i = 0; i < g[cur].e.size(); ++i) {
        const int v = g[cur].e[i];
        g[v].deg_in --;
        if (g[v].deg_in == 0) {
          s.push_back(v);
        }
      }
    }
    return res;
  }
\end{lstlisting}

\subsubsection{Floyd Warshall}
\begin{lstlisting}
inline vector<vector<int>> 
createFloydWarsharMatrix(int v) {
  auto result = 
    vector<vector<int>>{v, vector<int>(v, INT_MAX)};
  for (int i = 0; i < v; ++i) {
    result[i][i] = 0;
  }
  return result;
}
void floydWarshall(vector<vector<int>> & dist) {
  const int n = dist.size();
  for (int k = 0; k < n; ++k) {
    for (int i = 0; i < n; ++i) {
      for (int j = 0; j < n; ++j) {
        if (dist[i][k] == INT_MAX || 
            dist[k][j] == INT_MAX) 
              continue;
        dist[i][j] = 
          min(dist[i][k] + dist[k][j], 
              dist[i][j]);
      }
    }
  }
}
\end{lstlisting}

\subsubsection{Maximální párování}
\lstinputlisting{../Tung/hopcroft-karp.cpp}

\subsection{Stromové algoritmy}
\subsubsection{LCA}
\begin{lstlisting}
void treeLCA(int v = 0, 
             vector<Vertex> & g,
             UnionFind & uf,
             vector<int> & results,
             int depth = 0) {
  g[v].v = true;
  g[v].depth = depth;

  for (int i = 0; i < g[v].query.size(); ++i) {
    const int u = g[v].query[i].second;
    const int resultIndex = g[v].query[i].first;
    if (g[u].v) {
      // vypocita vzdalenost dvou vrcholu (pocet hran cesty)
      // d(v, u) = u.depth + v.depth - 2*lca(u, v).depth
      results[resultIndex] = 
        g[u].depth + g[v].depth - 2*g[ uf.root(u) ].depth;

      // vypocita nejblizsiho spolecneho predka
      //results[resultIndex] = uf.root(u);
    }
  }

  for (int i = 0; i < g[v].e.size(); ++i) {
    if (!g[ g[v].e[i] ].v) {
      treeLCA(g[v].e[i], g, uf, results, depth + 1);
      uf.parent[g[v].e[i]] = v;
    }
  }
}
// pridani dotazu pro treeLCA. 
// Vraci index, na kterem se bude nachazet vysledek
// v poli results.
int addTreeLCAQuery(
  vector<Vertex> & g, vector<int> & results, int v, int u) {
  g[v].query.push_back({results.size(), u});
  g[u].query.push_back({results.size(), v});
  const int r = (int) results.size();
  results.push_back(0);
  return r;
}
\end{lstlisting}

\subsubsection{Nejdelší ceste ve stromě}
\begin{lstlisting}
  void treeDiameterRec(
      vector<Vertex> & g,
      int n,
      int depth,
      int & maxD,
      int & maxDIdx) {
    if (g[n].v) return;
    g[n].v = true;
    g[n].depth = depth;
    if (depth > maxD) {
      maxD = depth;
      maxDIdx = n;
    }

    for (int i = 0; i < g[n].e.size(); ++i) {
      treeDiameterRec(g, g[n].e[i], depth + 1, maxD, maxDIdx);
    }
  }

  int startOfLongestPath, endOfLongestPath;

  // spocte sirku stromu (delku nejdelsi cesty)
  int treeDiameter(vector<Vertex> & g) {
    int maxD = 0, maxDIdx;
    treeDiameterRec(g, 0, 0, maxD, maxDIdx);
    for (int i = 0; i < g.size(); ++i) g[i].v = false;
    startOfLongestPath = maxDIdx;

    maxD = 0;
    treeDiameterRec(g, maxDIdx, 0, maxD, maxDIdx);
    for (int i = 0; i < g.size(); ++i) g[i].v = false;
    endOfLongestPath = maxDIdx;

    return maxD;
  }
\end{lstlisting}

\subsubsection{Střed stromu}
\begin{lstlisting}
  pair<bool, int> treeCenterRec(
  vector<Vertex> & g,
  int n, int depth,
  int diameter, int end) {
    if (g[n].v) return {false, -1};
    g[n].v = true;
    g[n].depth = depth;

    bool foundEnd = false;
    int endIdx = -1;
    if (n == end) {
      foundEnd = true;
    }

    for (int i = 0; i < g[n].e.size(); ++i) {
      auto a = 
        treeCenterRec(g, g[n].e[i], depth + 1, diameter, end);
      foundEnd |= a.first;
      if (a.second != -1) {
        endIdx = a.second;
        break;
      }
    }

    if (foundEnd && depth == diameter/2) {
      endIdx = n;
      return {true, endIdx};
    }

    return {foundEnd, endIdx};
  }

  // nalezne nejaky stredu stromu
  // O(n)
  int treeCenter(vector<Vertex> & g) {
    int diameter = treeDiameter(g);
    int result = 
      treeCenterRec(g, startOfLongestPath,
                    0, diameter, endOfLongestPath).second;
    for (int i = 0; i < g.size(); ++i) g[i].v = false;
    return result;
  }
\end{lstlisting}


\subsection{Stringové algoritmy}
\subsubsection{Longest common substring}
Jenom delka.
\begin{lstlisting}
int LongestCommonSubstringLen(string & p, string & q) {
  int result = 0;
  vector<vector<int>> dp(p.size()+1, vector<int>(q.size()+1, 0));
  for (int i = 1; i <= p.size(); ++i) {
    for (int j = 1; j <= q.size(); ++j) {
      if (p[i - 1] == q[j - 1]) {
        dp[i][j] = dp[i - 1][j - 1] + 1;
        result = max(result, dp[i][j]);
      }
    }
  }
  return result;
}
\end{lstlisting}

\subsubsection{Longest common subsequence}
Jenom delka.
\begin{lstlisting}
int LongestCommonSubsubsequenceLen(string & p, string & q) {
   int result = 0;
   vector<vector<int>> dp(p.size()+1, vector<int>(q.size()+1, 0));
   for (int i = 1; i <= p.size(); ++i) {
     for (int j = 1; j <= q.size(); ++j) {
       if (p[i - 1] == q[j - 1]) {
           dp[i][j] = dp[i - 1][j - 1] + 1;
           result = max(result, dp[i][j]);
       }
       dp[i][j] = max(dp[i][j], max(dp[i - 1][j], dp[i][j - 1]));
     }
   }
   return result;
}
\end{lstlisting}

\subsection{Mergesort s inverzemi}
\begin{lstlisting}
#include <iostream>
#include <vector>
#include <climits>
#include <cstring>
#include <algorithm>

/* Mergesort ktery dokaze spocitat pocet inverzi v permutaci.
 * To je, suma(x z Prvky)(f(x))
 * kde f(x) je pocet prvku ktere jsou pred x a jsou vyssi */

using namespace std;

long long inv;

int mergesort(vector<int> &src, int left, int right, int *result) {
  if (right == left) {
    result[0] = src[left];
    return 0;
  }

  int middle = (left + right) / 2;
  int resultL[middle - left + 2], resultR[right - middle + 1];
  resultL[middle - left + 1] = INT_MAX;
  resultR[right - middle] = INT_MAX;

  mergesort(src, left, middle, resultL);
  mergesort(src, middle + 1, right, resultR);

  // merge
  int l = 0, r = 0;
  for (int i = 0; i <= right - left; ++i) {
    if (resultL[l] < resultR[r]) {
      result[i] = resultL[l]; ++ l;
    } else {
      result[i] = resultR[r]; ++ r;
      inv += (middle - left + 1) - l;
    }
  }
  return 0;
}

int result[200000];

int main() {
  vector<int> input;

  // nacitani vstupu. Prvni na vstupu je pocet hodnot
  int N;
  cin >> N;

  for (int i = 0; i < N; ++i) {
    int n;
    cin >> n;
    input.push_back(n);
  }

  // pred zavolanim nad novou posloupnosti potreba nastavit inv = 0
  mergesort(input, 0, input.size() - 1, result);


  cout << "inversions: " << inv << endl;
  cout << "sorted output: " << endl;
  for (int i = 0; i < N - 1; ++i) {
    cout << result[i] << " ";
  }
  cout << result[N-1] << endl;


  return 0;
}
\end{lstlisting}

\subsection{Binární vyhledávání}
\begin{lstlisting}
  ll fn(ll i) {
    //if (i < 0) return INT_MIN;
    //if (i >= a.size()) return INT_MAX;
    //return a[i];
  }
\end{lstlisting}

\subsubsection{Upper bound}
Nejpravejsi roven
\begin{lstlisting}
  pair<ll,int> rightmostFind(int l, int r, ll val) {
    if (r >= l) {
      const int mid = (r + l) / 2;

      ll res = fn(mid);
      ll resNext = fn(mid + 1);

      if (res <= val && resNext > val) {
        return {res, mid};
      } else if (res > val) {
        return rightmostFind(l, mid - 1, val);
      } else {
        return rightmostFind(mid + 1, r, val);
      }
    }
    // nenalezeno
    return {0, -1};
  }
\end{lstlisting}

\subsubsection{Any bound}
Jakýkoliv roven
\begin{lstlisting}
  pair<ll,int> binarySearch(int l, int r, ll val) {
    if (r >= l) {
      const int mid = (r + l) / 2;

      ll res = fn(mid);

      if (res == val) {
        return {res, mid};
      } else if (res > val) {
        return binarySearch(l, mid - 1, val);
      } else {
        return binarySearch(mid + 1, r, val);
      }
    }
    // nenalezeno
    return {0, -1};
  };
\end{lstlisting}

\subsubsection{První větší}
\begin{lstlisting}
  pair<ll,int> upperbound(int l, int r, ll val) {
    if (r >= l) {
      const int mid = (r + l) / 2;

      ll res = fn(mid);
      ll resprev = fn(mid - 1);

      if (res > val && resprev <= val) {
        return {res, mid};
      } else if (res > val) {
        return upperbound(l, mid - 1, val);
      } else {
        return upperbound(mid + 1, r, val);
      }
    }
    // nenalezeno
    return {0, -1};
  };
\end{lstlisting}

\subsubsection{Lower bound}
Prvni >= val
\begin{lstlisting}
  pair<ll,int> lowerbound(int l, int r, ll val) {
    if (r >= l) {
      const int mid = (r + l) / 2;

      ll res = fn(mid);
      ll resprev = fn(mid - 1);

      if (res >= val && resprev < val) {
        return {res, mid};
      } else if (res >= val) {
        return lowerbound(l, mid - 1, val);
      } else {
        return lowerbound(mid + 1, r, val);
      }
    }
    // nenalezeno
    return {0, -1};
  };
\end{lstlisting}

\newpage

\section{Datové struktury}

\subsection{Big Integer}
\begin{lstlisting}
struct BigInt {
  typedef int NUM;
  vector<NUM> n; // change to char for memory efficiency
  int base = 10;

  BigInt() = default;
  BigInt(int m): n(vector<NUM>()) {
    if (m == 0) return;

    int sign = (m >= 0) ? 1 : -1;
    if (m < 0) m *= -1;
    while (m > 0) {
      n.push_back(m % base);
      m /= base;
    }
    fixRep();
    if (sign == -1) {
      *this = - *this;
    }
  }
  BigInt(const string & a) {
    if (a.size() == 0) return;
    if (a[0] == '-') {
      n = vector<NUM>(a.size() - 1);
      fromString(a, 1);
      *this = -*this;
    } else {
      n = vector<NUM>(a.size());
      fromString(a, 0);
    }
  }

  void fromString(const string & a, int start) {
    for (int i = 0; i < a.size() - start; ++i) {
      n[i] = a[a.size() - 1 - i] - '0';
    }
  }

  // addition
  void operator += (const BigInt & r) {
    if (r.size() > size())
      n.resize(r.size(), 0);

    for (int i = 0; i < size(); ++i) n[i] += r.n[i];

    fixNegatives();
    fixRep();
  }

  BigInt operator - () const {
    BigInt result = *this;
    for (int i = 0; i < result.size(); ++i) result.n[i] *= -1;
    result.fixNegatives();
    return result;
  }

  void operator -= (const BigInt & r) {
    *this += (- r);
  }

  BigInt operator + (const BigInt & r) const {
    BigInt result;
    result = *this;
    result += r;
    return result;
  }

  BigInt operator - (const BigInt & r) const {
    BigInt result = *this;
    result -= r;
    return result;
  }

  // multiplication
  void operator *= (const BigInt & r) {
    if (*this == 0 || r == 0) {
      *this = 0;
      return;
    }
    BigInt m;
    bool neg = (r < 0 != *this < 0);
    if (*this < 0) *this = -*this;
    if (r < 0) m = -r;
    else m = r;

    vector<vector<NUM>> matrix = 
    vector<vector<NUM>>(size(), 
      vector<NUM>(size() + m.size() + 1, 0));

    for (int i = 0; i < size(); ++i) {
      for (int j = 0; j < m.size(); ++j) {
        matrix[i][j + i] = n[i] * m.n[j];
      }
      // fix the base representation
      for (int j = 0; j < m.size() + 1; ++j) {
        matrix[i][j + i + 1] += matrix[i][j + i]/base;
        matrix[i][j + i] %= base;
      }
    }
    // sum together
    for (int i = 1; i < size(); ++i) {
      for (int j = 0; j < size() + m.size() + 1; ++j) {
        matrix[i][j] += matrix[i - 1][j];
      }
    }
    const int s = size();
    n.resize(size() + m.size() + 2, 0);
    for (int i = 0; i < s + m.size() + 1; ++i) {
      n[i] = matrix[s - 1][i];
    }
    fixRep();

    if (neg) *this = -*this;
  }

  BigInt operator * (const BigInt & r) const {
    BigInt result = *this;
    result *= r;
    return result;
  }

  // comparision
  bool operator == (const BigInt & r) const {
    if (r.size() != size()) return false;
    for (int i = 0; i < r.size(); ++i) {
      if (r.n[i] != n[i]) return false;
    }
    return true;
  }

  bool operator < (const BigInt & r) const {
    if (isNegative() && !r.isNegative()) return true;
    else if (!isNegative() && r.isNegative()) return false;
    else if (isNegative() && r.isNegative()) {
      return (-r < -*this);
    }
    if (r.size() > size()) return true;
    else if (r.size() < size()) return false;

    for (int i = r.size() - 1; i >= 0; --i) {
      if (r.n[i] > n[i]) return true;
      else if (r.n[i] < n[i]) return false;
    }
    return false;
  }

  bool operator > (const BigInt & r) const {
    return r < *this;
  }

  bool operator <= (const BigInt & r) const {
    return !(*this > r);
  }

  bool operator >= (const BigInt & r) const {
    return !(*this < r);
  }

  // O(n) kde n je pocet cifer
  void leftShift() {
    n.insert(n.begin(), 0);
  }

  bool isNegative() const {
    return (size() > 0 && n[size() - 1] < 0);
  }

  void fixNegatives() {
    for (int i = 0; i < size() - 1; ++i) {
      if (n[i] < 0) {
        n[i] += base;
        n[i + 1] --;
      }
    }
  }

  void fixRep() {
    for (int i = 1; i < size(); ++i) {
      n[i] += n[i - 1]/base;
      n[i - 1] %= base;
    }

    if (size() >= 1 && abs(n[size() - 1]) >= base) {
      n.push_back(n[size() - 1]/base);
      n[size() - 2] %= base;
    }
    // removed excess zeroes
    while (size() > 0 && n[size() - 1] == 0) n.pop_back();
  }

  int size() const { return (int) n.size(); }

  operator string() const {
    if (size() == 0) { return "0"; }
    string result;
    if (n[size() - 1] >= 0) {
      for (int i = size() - 1; i >= 0; --i) {
        result.push_back(n[i] + '0');
      }
    } else {
      result = "-" + (string)(- *this);
    }
    return result;
  }
};

\end{lstlisting}

\subsection{Fenwick}
\begin{lstlisting}
#define vi vector< int >

using namespace std;

vi T;

void add( int i, int delta )
{
  while( i < T.size( ) )
  {
    T[ i ] += delta;
    i += ( i & -i );
  }
}

int pref_sum( int i )
{
  int res = 0;
  while( i > 0 )
  {
    s += T[ i ];
    i = i & ( i - 1 );
  }

  return res;
}
\end{lstlisting}

\subsection{Union Find}
\begin{lstlisting}
#include <vector>
using namespace std;
struct UnionFind {
  vector<int> parent;
  vector<int> depth;
  UnionFind(int n): parent(vector<int>(n)), 
                    depth(vector<int>(n, 0)) {
    for (int i = 0; i < n; ++ i) parent[i] = i;
  }

  // nalezne koren pro vrchol v
  // O(log*(n))
  int root(int v) {
    return (v == parent[v]) ? v : parent[v] = root(parent[v]);
  }

  // zjisti, jestli jsou dva vektory ve stejne komponente
  bool find(int a, int b) {
    return root(a) == root(b);
  }

  // spoji dva vrcholy do jedne komponenty
  void do_union(int a, int b) {
    if (a == b) return;
    if (depth[a] < depth[b]) {
      parent[a] = b;
    } else if (depth[a] > depth[b]) {
      parent[b] = a;
    } else {
      parent[b] = a;
      depth[a] ++;
    }
  }
};
\end{lstlisting}

\newpage
\section{Matematika}

\subsection{Typedefy}
\begin{lstlisting}
typedef unsigned long long ull;
typedef long long ll;
typedef pair<int,int> P;
using namespace std;
\end{lstlisting}

\subsection{Struktury}
\subsubsection{Matice}
\begin{lstlisting}
struct Matrix {
  Matrix(int rows, int columns):
      m(vector<vector<ll>>{rows, vector<ll>(columns, 0)})
      , rows(rows)
      , columns(columns) {
  }
  Matrix(int size, bool unit = false):
      m(vector<vector<ll>>{size, vector<ll>(size, 0)})
    , rows(size), columns(size) {
    if (unit) {
      for (int i = 0; i < size; ++i) {
        m[i][i] = 1;
      }
    }
  }

  Matrix operator * (const Matrix & other) const {
    Matrix result(rows, columns);
    for (int i = 0; i < 2; ++i) {
      for (int j = 0; j < 2; ++j) {
        for (int k = 0; k < 2; ++k) {
          result.m[i][j] = 
            d(d(result.m[i][j]) + 
              d(d(m[i][k]) * d(other.m[k][j])));
        }
      }
    }
    return result;
  }

  vector<vector<ll>> m;
  int rows, columns;
};
\end{lstlisting}

\subsection{Odčítání v modulu}
Proprietary (x - y) \% k
\begin{lstlisting}
inline ll modsub( ll x, ll y, ll k )
{
  return ( x % k - y % k + k ) % k;
}
\end{lstlisting}

\subsection{GCD}
\begin{lstlisting}
  // nejmensi spolecny delitel
  ll gcd(ll m, ll n) {
    if(m == 0 && n == 0)
      return -1;

    if(m < 0) m = -m;
    if(n < 0) n = -n;

    ll r;
    while(n) {
      r = m % n;
      m = n;
      n = r;
    }
    return m;
  }
\end{lstlisting}

\subsection{Kombinačni číslo}
\begin{lstlisting}
  ull NchooseK(ull n, ull k)
  {
    if (k == 0) return 1;
    if (k < n) return 0;

    if (k > n/2) return NchooseK(n, n-k);

    ull out = 1;
    for(int k = 1; k <= k; ++k) {
      out *= n-k+1;
      out /= k;
    }

    return out;
  }
\end{lstlisting}

\subsection{Modularní inverze}
\begin{lstlisting}
ll modularInversion(ll n, ll m) {
  ll mod = m;
  if (n < 0) n = m + n;
  n %= m;

  ll p[2], q[2], d[2];
  p[0] = 1; p[1] = 0;
  q[0] = 0; q[1] = 1;
  d[0] = 0; d[1] = m / n;

  while(n) {
    ll tp, tq, td, tn;
    tp = p[0] - d[1]*p[1];
    tq = q[0] - d[1]*q[1];
    tn = m - d[1]*n;
    if (tn == 0) break;
    td = n / tn;

    // move to the next line
    m = n;
    n = tn;
    p[0] = p[1];
    p[1] = tp;
    q[0] = q[1];
    q[1] = tq;
    d[0] = d[1];
    d[1] = td;
  }

  if (q[1] < 0) {
    q[1] += mod * ((q[1] / mod) + 1);
  }

  return q[1];
}
\end{lstlisting}

\subsection{Kombinacni cislo modulené(?)}
\begin{lstlisting}
  ull NchooseKmodP(ull n, ull k, ull p) {
    ull top = 1;
    ull bottom = 1;
    for (ull i = n; i > k; -- i) {
      top *= i;
      top %= p;
    }
    for (ull i = n-k; i > 0; -- i) {
      bottom *= i;
      bottom %= p;
    }

    return ((top % p) * (modularInversion(bottom, p) % p)) % p;
  }
\end{lstlisting}

\subsection{Eratosthenova síta}
\begin{lstlisting}
  vector<int> getPrimes(int upTo) {
    vector<int> result = {2, 3, 5};
    for (int i = 6; i <= upTo; ++i) {
      bool isPrime = true;
      for (int j = 0; 
           j < result.size() && result[j] * result[j] <= i; 
           ++ j) {
        if (i % result[j] == 0) {
          isPrime = false;
        }
      }
      if (isPrime)
        result.push_back(i);
      }
    return result;
  }

  vector<int> erasothenes(int upTo) {
    vector<bool> sieve(upTo + 1, false);
    for (int i = 2; i <= upTo; ++i) {
      if (!sieve[i]) {
        for (int j = 2*i; j <= upTo; j += i) {
          sieve[j] = true;
        }
      }
    }

    vector<int> result;
    for (int i = 2; i <= upTo; ++i) {
      if (!sieve[i])
        result.push_back(i);
      }
    return result;
  }
\end{lstlisting}

\subsection{Faktorizace}
\begin{lstlisting}
  vector<pair<int, int>> 
  factors(int n, const vector<int> & primes) {
    vector<pair<int, int>> result;
    for (int i = 0; i < primes.size() && primes[i] <= n; ++i) {
      int c = 0;
      while (n % primes[i] == 0) {
        c ++;
        n /= primes[i];
      }
      if (c != 0) {
        result.push_back({primes[i], c});
      }
    }
    return result;
  }
\end{lstlisting}

\subsection{Počet dělitelů (potřeba faktorizace výše!)}
\begin{lstlisting}
  int divisorsCount(const vector<pair<int, int>> &facts) {
    int d = 1;
    for (int i = 0; i < facts.size(); ++i) {
      d *= (facts[i].second + 1);
    }
    return d;
  }
\end{lstlisting}

\subsection{Počet čtvercových dělitelů}
\begin{lstlisting}
  int squareDivisors(vector<pair<int, int>> facts) {
    bool confirm = false;
    // set to false if not counting proper divisorsCount
    bool isEvenPerfectSquare = true;

    for (int i = 0; i < facts.size(); ++i) {
      if (facts[i].second % 2 != 0) isEvenPerfectSquare = false;
      if (facts[i].first == 2 && facts[i].second >= 2) {
        confirm = true;
        facts[i].second -= 2;
      }
      facts[i].second /= 2;
    }

    int subtract = (confirm && isEvenPerfectSquare) ? 1 : 0;

    if (confirm)
      return divisorsCount(facts) - subtract;
    else
      return 0;
    }
\end{lstlisting}

\subsection{Rychlé umocňování čísel}
\begin{lstlisting}
  int ipow(int base, int exp) {
    int result = 1;
      while (exp)
      {
        if (exp & 1) {
            result *= base;
            // modulit tady
          }
        exp >>= 1;
        base *= base;
        // modulit tady
      }

      return result;
    }
\end{lstlisting}

\subsection{Rychlé umocňování matic}
\begin{lstlisting}
  Matrix ipow(Matrix base, ll exp) {
    Matrix result(base.rows, true);
    while (exp)
    {
      if (exp & 1)
          result = result * base;
      exp >>= 1;
      base = base * base;
    }

    return result;
  }
\end{lstlisting}

\subsection{Rozšířené Eratosthenovo síto}
Faktorizace v O(log(n))
\begin{lstlisting}
  vector<int> erasothenesExt(int upTo) {
    vector<bool> v(upTo + 1, false);
    vector<int> sp(upTo + 1, 0);
    for (int i = 2; i <= upTo; ++i) {
      if (!v[i]) {
        for (int j = 2*i; j <= upTo; j += i) {
          if (!v[j]) {
            v[j] = true;
            sp[j] = i;
          }
        }
        sp[i] = i;
      }
    }
    sp[1] = 1;
    return sp;
  }
\end{lstlisting}

// rychla faktorizace v O(log(n)) za pomoci rozsireneho sita
\subsection{Rychlá faktorizace}
Potreba rozsirene Eratosthenovo sito vyse
\begin{lstlisting}
  vector<P> fastFactors(int n, vector<int> & sieve) {
    int c;
    vector<P> divs;
    divs.push_back({sieve[n], 1});
    n /= sieve[n];
    while (n != 1) {
      c = sieve[n];
      if (divs.back().first != c) {
        divs.push_back({c, 1});
      } else {
        divs.back().second ++;
      }
      n = n/sieve[n];
    }
    return divs;
  }
\end{lstlisting}

\subsection{Fast divisors}
At je to cokoliv

\begin{lstlisting}
  // pomocna funkce pro fastDivisors
  void getDivs(vector<int> & res, 
               vector<P> & divs,
               int t, int i) {
    if (i >= divs.size()) {
      res.push_back(t);
      return;
    }
    int n = t;
    for (int j = 0; j <= divs[i].second; ++j) {
      getDivs(res, divs, n, i + 1);
      n *= divs[i].first;
    }
  }

  // ziska delitele v asymptoticky optimalnim case
  vector<int> fastDivisors(int n, vector<int> sieve) {
    vector<P> divs = fastFactors(n, sieve);
    vector<int> exp(divs.size(), 0);
    vector<int> result;

    getDivs(result, divs, 1, 0);
    return result;
  }
\end{lstlisting}
\end{document}

\subsection{Minimal subarray sum}
Nejkratsi usek pole roven nejakemu k.

\begin{lstlisting}
  int minSubarraySum(vector<int> ar, int k) {
    if (k == 0) return 0;
    int minLen = INT_MAX;
    int start = 0, end = 0;
    int curSum = ar[0];
    while (end != ar.size() - 1 || start != ar.size() - 1) {
      if (curSum == k) minLen = min(minLen, end - start + 1);

      if (curSum >= k && start < end) {
        curSum -= ar[start];
        start++;
      } else {
        if (end < ar.size() - 1) {
          end++;
          curSum += ar[end];
        } else {
          break;
        }
      }
    }
    if (curSum == k) minLen = min(minLen, end - start + 1);
    return minLen;
  }
\end{lstlisting}
