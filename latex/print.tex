\documentclass{article}
\usepackage{listings}
\usepackage[czech]{babel}
\usepackage[utf8]{inputenc}

% nastavuje code blocks
\lstset{
  language=C++,
  numbers=none
}

\title{Knihovna tymu My z Kacerova}
\author{Matyas Kristan, Vu Tung Anh}

\begin{document}
\maketitle
\pagenumbering{gobble}
\newpage
\pagenumbering{arabic}


\section{Algoritmy}
\subsection{Grafove algoritmy}
\subsubsection{Reprezentace}
\begin{lstlisting}
struct Vertex {
  vector<int> e;
  int deg_in;
  bool visited = false;
  int depth;
  vector<P> query;
};
\end{lstlisting}

\subsubsection{Topsort}
\begin{lstlisting}
  vector<int> topsort(vector<Vertex> & g) {
    vector<int> res;
    vector<int> s;
    for (int i = 0; i < g.size(); ++i) {
      if (g[i].deg_in == 0) s.push_back(i);
    }
    while (!s.empty()) {
      const int cur = s.back();
      s.pop_back();
      res.push_back(cur);

      for (int i = 0; i < g[cur].e.size(); ++i) {
        const int v = g[cur].e[i];
        g[v].deg_in --;
        if (g[v].deg_in == 0) {
          s.push_back(v);
        }
      }
    }
    return res;
  }
\end{lstlisting}

\subsubsection{Floyd Warshall}
\begin{lstlisting}
inline vector<vector<int>> 
createFloydWarsharMatrix(int v) {
  auto result = 
    vector<vector<int>>{v, vector<int>(v, INT_MAX)};
  for (int i = 0; i < v; ++i) {
    result[i][i] = 0;
  }
  return result;
}
void floydWarshall(vector<vector<int>> & dist) {
  const int n = dist.size();
  for (int k = 0; k < n; ++k) {
    for (int i = 0; i < n; ++i) {
      for (int j = 0; j < n; ++j) {
        if (dist[i][k] == INT_MAX || 
            dist[k][j] == INT_MAX) 
              continue;
        dist[i][j] = 
          min(dist[i][k] + dist[k][j], 
              dist[i][j]);
      }
    }
  }
}
\end{lstlisting}

\subsubsection{LCA}
\begin{lstlisting}
void treeLCA(int v = 0, 
             vector<Vertex> & g,
             UnionFind & uf,
             vector<int> & results,
             int depth = 0) {
  g[v].visited = true;
  g[v].depth = depth;

  for (int i = 0; i < g[v].query.size(); ++i) {
    const int u = g[v].query[i].second;
    const int resultIndex = g[v].query[i].first;
    if (g[u].visited) {
      // vypocita vzdalenost dvou vrcholu (pocet hran cesty)
      // d(visited, u) = u.depth + v.depth - 2*lca(u, visited).depth
      results[resultIndex] = g[u].depth + g[v].depth - 2*g[ uf.root(u) ].depth;

      // vypocita nejblizsiho spolecneho predka
      //results[resultIndex] = uf.root(u);
    }
  }

  for (int i = 0; i < g[v].e.size(); ++i) {
    if (!g[ g[v].e[i] ].visited) {
      treeLCA(g[v].e[i], g, uf, results, depth + 1);
      uf.parent[g[v].e[i]] = v;
    }
  }
}
// pridani dotazu pro treeLCA. Vraci index, na kterem se bude nachazet vysledek
// v poli results.
int addTreeLCAQuery(vector<Vertex> & g, vector<int> & results, int v, int u) {
  g[v].query.push_back({results.size(), u});
  g[u].query.push_back({results.size(), v});
  const int r = (int) results.size();
  results.push_back(0);
  return r;
}
\end{lstlisting}

\subsection{Stringove algoritmy}
\subsubsection{Longest common substring}
Jenom delka.
\begin{lstlisting}
int LongestCommonSubstringLen(string & p, string & q) {
  int result = 0;
  vector<vector<int>> dp(p.size()+1, vector<int>(q.size()+1, 0));
  for (int i = 1; i <= p.size(); ++i) {
    for (int j = 1; j <= q.size(); ++j) {
      if (p[i - 1] == q[j - 1]) {
        dp[i][j] = dp[i - 1][j - 1] + 1;
        result = max(result, dp[i][j]);
      }
    }
  }
  return result;
}
\end{lstlisting}

\subsection{Mergesort s inverzemi}
\begin{lstlisting}
//
// Created by maty on 3/3/17.
//

#include <iostream>
#include <vector>
#include <climits>
#include <cstring>
#include <algorithm>

/* Mergesort ktery dokaze spocitat pocet inverzi v permutaci.
 * To je, suma(x z Prvky)(f(x))
 * kde f(x) je pocet prvku ktere jsou pred x a jsou vyssi */

using namespace std;

long long inv;

int mergesort(vector<int> &src, int left, int right, int *result) {
  if (right == left) {
    result[0] = src[left];
    return 0;
  }

  int middle = (left + right) / 2;
  int resultL[middle - left + 2], resultR[right - middle + 1];
  resultL[middle - left + 1] = INT_MAX;
  resultR[right - middle] = INT_MAX;

  mergesort(src, left, middle, resultL);
  mergesort(src, middle + 1, right, resultR);

  // merge
  int l = 0, r = 0;
  for (int i = 0; i <= right - left; ++i) {
    if (resultL[l] < resultR[r]) {
      result[i] = resultL[l]; ++ l;
    } else {
      result[i] = resultR[r]; ++ r;
      inv += (middle - left + 1) - l;
    }
  }
  return 0;
}

int result[200000];

int main() {
  vector<int> input;

  // nacitani vstupu. Prvni na vstupu je pocet hodnot
  int N;
  cin >> N;

  for (int i = 0; i < N; ++i) {
    int n;
    cin >> n;
    input.push_back(n);
  }

  // pred zavolanim nad novou posloupnosti potreba nastavit inv = 0
  mergesort(input, 0, input.size() - 1, result);


  cout << "inversions: " << inv << endl;
  cout << "sorted output: " << endl;
  for (int i = 0; i < N - 1; ++i) {
    cout << result[i] << " ";
  }
  cout << result[N-1] << endl;


  return 0;
}
\end{lstlisting}


\section{Datové struktury}
\subsection{Fenwick}
\begin{lstlisting}
#define vi vector< int >

using namespace std;

vi T;

void add( int i, int delta )
{
  while( i < T.size( ) )
  {
    T[ i ] += delta;
    i += ( i & -i );
  }
}

int pref_sum( int i )
{
  int res = 0;
  while( i > 0 )
  {
    s += T[ i ];
    i = i & ( i - 1 );
  }

  return res;
}
\end{lstlisting}

\subsection{Union Find}
\begin{lstlisting}
#include <vector>
using namespace std;
struct UnionFind {
  vector<int> parent;
  vector<int> depth;
  UnionFind(int n): parent(vector<int>(n)), depth(vector<int>(n, 0)) {
    for (int i = 0; i < n; ++ i) parent[i] = i;
  }

  // nalezne koren pro vrchol v
  // O(log*(n))
  int root(int v) {
    return (v == parent[v]) ? v : parent[v] = root(parent[v]);
  }

  // zjisti, jestli jsou dva vektory ve stejne komponente
  bool find(int a, int b) {
    return root(a) == root(b);
  }

  // spoji dva vrcholy do jedne komponenty
  void do_union(int a, int b) {
    if (a == b) return;
    if (depth[a] < depth[b]) {
      parent[a] = b;
    } else if (depth[a] > depth[b]) {
      parent[b] = a;
    } else {
      parent[b] = a;
      depth[a] ++;
    }
  }
};
\end{lstlisting}

\section{Matematika}

\subsection{Typedefy}
\begin{lstlisting}
typedef unsigned long long ull;
typedef long long ll;
typedef pair<int,int> P;
using namespace std;
\end{lstlisting}

\subsection{GCD}
\begin{lstlisting}
  // nejmensi spolecny delitel
  ll gcd(ll m, ll n) {
    if(m == 0 && n == 0)
      return -1;

    if(m < 0) m = -m;
    if(n < 0) n = -n;

    ll r;
    while(n) {
      r = m % n;
      m = n;
      n = r;
    }
    return m;
  }
\end{lstlisting}

\subsection{Kombinacni cislo}
\begin{lstlisting}
  ull NchooseK(ull n, ull k)
  {
    if (k == 0) return 1;
    if (k < n) return 0;

    if (k > n/2) return NchooseK(n, n-k);

    ull out = 1;
    for(int k = 1; k <= k; ++k) {
      out *= n-k+1;
      out /= k;
    }

    return out;
  }
\end{lstlisting}

\subsection{Modularni inverze}
\begin{lstlisting}
ll modularInversion(ll n, ll m) {
  ll mod = m;
  if (n < 0) n = m + n;
  n %= m;

  ll p[2], q[2], d[2];
  p[0] = 1; p[1] = 0;
  q[0] = 0; q[1] = 1;
  d[0] = 0; d[1] = m / n;

  while(n) {
    ll tp, tq, td, tn;
    tp = p[0] - d[1]*p[1];
    tq = q[0] - d[1]*q[1];
    tn = m - d[1]*n;
    if (tn == 0) break;
    td = n / tn;

    // move to the next line
    m = n;
    n = tn;
    p[0] = p[1];
    p[1] = tp;
    q[0] = q[1];
    q[1] = tq;
    d[0] = d[1];
    d[1] = td;
  }

  if (q[1] < 0) {
    q[1] += mod * ((q[1] / mod) + 1);
  }

  return q[1];
}
\end{lstlisting}

\subsection{Kombinacni cislo modulene(?)}
\begin{lstlisting}
  ull NchooseKmodP(ull n, ull k, ull p) {
    ull top = 1;
    ull bottom = 1;
    for (ull i = n; i > k; -- i) {
      top *= i;
      top %= p;
    }
    for (ull i = n-k; i > 0; -- i) {
      bottom *= i;
      bottom %= p;
    }

    return ((top % p) * (modularInversion(bottom, p) % p)) % p;
  }
\end{lstlisting}

// vrati vektor vsech prvocisel <= upTo
\subsection{Eratosthenova sita (ruzne rychla)}
\begin{lstlisting}
  vector<int> getPrimes(int upTo) {
    vector<int> result = {2, 3, 5};
    for (int i = 6; i <= upTo; ++i) {
      bool isPrime = true;
      for (int j = 0; 
           j < result.size() && result[j] * result[j] <= i; 
           ++ j) {
        if (i % result[j] == 0) {
          isPrime = false;
        }
      }
      if (isPrime)
        result.push_back(i);
      }
    return result;
  }

  vector<int> erasothenes(int upTo) {
    vector<bool> sieve(upTo + 1, false);
    for (int i = 2; i <= upTo; ++i) {
      if (!sieve[i]) {
        for (int j = 2*i; j <= upTo; j += i) {
          sieve[j] = true;
        }
      }
    }

    vector<int> result;
    for (int i = 2; i <= upTo; ++i) {
      if (!sieve[i])
        result.push_back(i);
      }
    return result;
  }
\end{lstlisting}

\subsection{Faktorizace}
\begin{lstlisting}
  vector<pair<int, int>> factors(int n, const vector<int> & primes) {
    vector<pair<int, int>> result;
    for (int i = 0; i < primes.size() && primes[i] <= n; ++i) {
      int c = 0;
      while (n % primes[i] == 0) {
        c ++;
        n /= primes[i];
      }
      if (c != 0) {
        result.push_back({primes[i], c});
      }
    }
    return result;
  }
\end{lstlisting}

\subsection{Pocet delitelu (potreba faktorizace vyse!)}
\begin{lstlisting}
  int divisorsCount(const vector<pair<int, int>> &facts) {
    int d = 1;
    for (int i = 0; i < facts.size(); ++i) {
      d *= (facts[i].second + 1);
    }
    return d;
  }
\end{lstlisting}

\subsection{Pocet ctvercovych delitelu}
\begin{lstlisting}
  int squareDivisors(vector<pair<int, int>> facts) {
    bool confirm = false;
    // set to false if not counting proper divisorsCount
    bool isEvenPerfectSquare = true;

    for (int i = 0; i < facts.size(); ++i) {
      if (facts[i].second % 2 != 0) isEvenPerfectSquare = false;
      if (facts[i].first == 2 && facts[i].second >= 2) {
        confirm = true;
        facts[i].second -= 2;
      }
      facts[i].second /= 2;
    }

    int subtract = (confirm && isEvenPerfectSquare) ? 1 : 0;

    if (confirm)
      return divisorsCount(facts) - subtract;
    else
      return 0;
    }
\end{lstlisting}

\subsection{Rychle umocnovani}
\begin{lstlisting}
  int ipow(int base, int exp) {
    int result = 1;
      while (exp)
      {
        if (exp & 1)
              result *= base;
          exp >>= 1;
          base *= base;
        }

      return result;
    }
\end{lstlisting}

\subsection{Rozsirene Eratosthenovo sito}
Faktorizace v O(log(n))
\begin{lstlisting}
  vector<int> erasothenesExt(int upTo) {
    vector<bool> v(upTo + 1, false);
    vector<int> sp(upTo + 1, 0);
    for (int i = 2; i <= upTo; ++i) {
      if (!v[i]) {
        for (int j = 2*i; j <= upTo; j += i) {
          if (!v[j]) {
            v[j] = true;
            sp[j] = i;
          }
        }
        sp[i] = i;
      }
    }
    sp[1] = 1;
    return sp;
  }
\end{lstlisting}

// rychla faktorizace v O(log(n)) za pomoci rozsireneho sita
\subsection{Rychla faktorizace}
Potreba rozsirene Eratosthenovo sito vyse
\begin{lstlisting}
  vector<P> fastFactors(int n, vector<int> & sieve) {
    int c;
    vector<P> divs;
    divs.push_back({sieve[n], 1});
    n /= sieve[n];
    while (n != 1) {
      c = sieve[n];
      if (divs.back().first != c) {
        divs.push_back({c, 1});
      } else {
        divs.back().second ++;
      }
      n = n/sieve[n];
    }
    return divs;
  }
\end{lstlisting}

\subsection{Fast divisors}
At je to cokoliv

\begin{lstlisting}
  // pomocna funkce pro fastDivisors
  void getDivs(vector<int> & res, 
               vector<P> & divs,
               int t, int i) {
    if (i >= divs.size()) {
      res.push_back(t);
      return;
    }
    int n = t;
    for (int j = 0; j <= divs[i].second; ++j) {
      getDivs(res, divs, n, i + 1);
      n *= divs[i].first;
    }
  }

  // ziska delitele v asymptoticky optimalnim case
  vector<int> fastDivisors(int n, vector<int> sieve) {
    vector<P> divs = fastFactors(n, sieve);
    vector<int> exp(divs.size(), 0);
    vector<int> result;

    getDivs(result, divs, 1, 0);
    return result;
  }
\end{lstlisting}
